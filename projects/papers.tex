\documentclass[12pt]{article}%
\usepackage{amssymb}
\usepackage{amsfonts}
\usepackage{amsmath}
\usepackage{amsthm}
\usepackage{fancyhdr}
\usepackage{parskip}


\usepackage{url}
\usepackage[nohead]{geometry}
\geometry{left=1in,right=1in,top=1.00in,bottom=1.0in}

\usepackage{graphicx}%
\usepackage[colorlinks=True,urlcolor=blue]{hyperref}

%\usepackage{natbib} % natbib is essentially incompatible with multibib
%\setcitestyle{numbers}
%\renewcommand{\cite}{\citet}
\usepackage[resetlabels,labeled]{multibib}
% biblio commands
\newcites{Easy}{\large Short and ``easy" papers}
\newcites{Hard}{\large Long or difficult papers}

\begin{document}

\title{Bayesian ML: project topics}
\author{R\'emi Bardenet and Julyan Arbel}
\maketitle

\section{Nature of the project}
A group is made of either one or two students.
Each group is to pick one of the papers listed below, and each paper can be chosen only once.
The assignment in on a first-come first-served basis.
Groups of two must choose a paper from the ``long/difficult" list. 
For groups of one, there will be bonus points for picking a paper identified as ``long/difficult'', but it will be possible to get the maximum grade with a paper identified as ``short/easy''.
You can also choose to work on a paper of your choice, subject to our explicit approval; in that case, we trust you not to pick a paper that you are or have been working on in another class.

The whole point is to read your paper with a critical mind. 
For the paper your group will have chosen, you should: (1) explain the contents of the paper, (2) emphasize the strong and weak points of the paper, and (3) apply it to real data of your choice when applicable. 
Bonus points will be considered if you are creative and add something insightful that is not in the original paper: this can be a theoretical point, an illustrative experiment, etc. 
Be explicit in your introduction what is your creative contribution.

\section{Assignment of papers}
As a first step, we ask each group to fill the spreadsheet at
\begin{center}
   \href{https://lite.framacalc.org/zr1v1h3ld6-9z79}{https://lite.framacalc.org/zr1v1h3ld6-9z79}
 \end{center}
with the title of the paper, a link to it (if available), and the composition of the group.
Please fill the form {\bf before Thursday 16 February}. 
%By that time, you will have had an outline of the last two courses, so that you can make your choices with enough information. Then, we will solve the \href{https://en.wikipedia.org/wiki/Assignment_problem}{assignment problem} to find an optimal matching of papers and groups, and will make the result known to all as soon as possible on the \href{Github}{https://github.io/bml-course} page of the course. We will allow up to two groups per paper, but in that case we expect of course the deliverables to be significantly different.

\section{Format of the deliverable}
Please have each group send
\begin{itemize}
\item one report as a pdf ($\leq 5$ pages) in the \href{https://www.overleaf.com/latex/templates/neurips-2020/mnshsmqkjsqz}{NeurIPS template},
\item the link to a \href{https://github.com/}{GitHub} or \href{https://about.gitlab.com/}{GitLab} repository containing your code and a detailed readme file with instructions to (compile/install and) run the code.
\end{itemize} to both \href{mailto:remi.bardenet@gmail.com}{remi.bardenet@gmail.com} and \href{mailto: julyan.arbel@inria.fr}{ julyan.arbel@inria.fr}{\bf no later than March 16th}. 
We will then meet on March 23rd for presentations (10+5 minutes per group).

\section{Proposed papers}
\label{s:papers}

\bibliographystyleEasy{plain}
\bibliographystyleHard{plain}

\nociteEasy{*}
\nociteHard{*}

\bibliographyEasy{easy.bib}
\bibliographyHard{hard.bib}
\end{document}
