In this chapter, we introduce different historical approaches to learning and inference on a running example, penalized linear regression. 
We shall exagerate the stances of some famous scientists, for the sake of illustration. 

\section{Fisher does linear regression}

Say we have a data set $(x_i,y_i)$, $1\leq i\leq N$, where $x_i\in\mathbb{R}^d$ are the \emph{features} and $y_i\in\mathbb{R}$ the \emph{response}. 
We want to study the influence of features on the response, and we ask British statistician Ronald Fisher (1890--1962) for help. 
He recommends positing a simple statistical model, i.e. a parametrized collection of PDFs for $y\vert x$. 
Calling the parameter $\theta$ and abusively denoting the parametrized distributions by $y\vert x,\theta$, we posit 
$$ p(y_i\vert x_i, \theta) = \cN(y_i \vert x_i^T \theta, \sigma^2), 1\leq i \leq N,$$
i.i.d., with known $\sigma$ for simplicity.
To characterize the influence of features on the response, Fisher recommends that we estimate $\theta$, along with a confidence interval to communicate our uncertainty. 
Fisher was the one to introduce the maximum likelihood estimator 
$$
\tMLE \triangleq \arg\max_{\theta} p(\by\vert X, \theta) = \arg\max_\theta \prod_{i=1}^N \cN(y_i \vert x_i^T \theta, \sigma^2).
$$
Now 
$$
\log \prod_{i=1}^N \cN(y_i \vert x_i^T \theta, \sigma^2) = \log \cN(\by \vert X\theta, \sigma^2 I) \propto - \Vert \by - X\theta\Vert^2,
$$
where the proportionality sign allows us to drop additive terms that do not depend on $\theta$. 
This entails that $\tMLE$ is nothing but the least squares estimator 
$$
\tMLE = (X^TX)^{-1} X^T\by,
$$
where we assumed that $X$ has full rank.
There is nothing inherently good in using the MLE rather than another estimator, but it often has good \emph{frequentist} properties, i.e., properties established by integrating over the posited data generation model.
For instance, it is easy to prove the following.
\begin{proposition}
Under $\by \sim \cN(X\theta, \sigma^2 I)$, and assuming $X$ has full rank,
\begin{equation}
    \label{e:fisher}
    \tMLE \sim \cN(\theta, \sigma^2(X^TX)^{-1}).
\end{equation}
\end{proposition}
\proofLeftAsExercise

Fisher then argues that \eqref{e:fisher} gives you a wealth of properties that make $\tMLE$ an interesting estimator.
For starters, $\tMLE$ is unbiased ($\mathbb E \tMLE = \theta$)
and $\tMLE \rightarrow \theta$ in probability. 
Moreover, the (random) ellipsoid 
$$
\mathcal{E}_\alpha = \{\theta\in\mathbb{R}^d \text{ such that }\sigma^{-2}(\theta-\tMLE)^TX^T X(\theta-\tMLE) \leq \alpha\} 
$$
contains $\theta$ with known probability $1-\delta(\alpha)$.
It thus makes sense, to Fisher, to find the smallest value $\alpha_{0.99}$ such that $1-\delta(\alpha)\geq 0.99$, and report the \emph{confidence region} $\mathcal{E}_{\alpha_0.99}$ to quantify his uncertainty about $\theta$. 
The property that, under repetition of the data generation, the (random) ellipsoid $\mathcal{E}_{\alpha_{0.99}}$ contains $\theta$ about $99\%$ of the time, is called \emph{coverage}.

\section{Wald does linear regression}
Imagine that you had asked Hungarian-American statistician Abraham Wald (1902--1950) for help instead of Fisher.
He would have argued that estimating $\theta$, or outputing a region of $\mathbb{R}^d$ that encodes your uncertainty, are two different actions to be taken under uncertainty. 

Consider estimation. 
Wald would ask you to specify the loss $L(\theta,\widehat{\theta})$ that you incur by estimating $\theta$ by $\widehat{\theta}\in\mathbb{R}^d$. 
You might answer $L_\alpha(\theta,\widehat{\theta}) = \Vert\theta-\widehat{\theta}\Vert^\alpha$ for some $\alpha>0$.
Wald would then say that the accuracy of an estimator $\widehat{\theta}= \widehat{\theta}(\by)$ is characterized by its risk function
\begin{equation}
    R(\cdot,\widehat{\theta}):\theta\mapsto \mathbb{E}_{\by\vert X, \theta} L(\theta,\widehat{\theta}(\by)).
    \label{e:risk}
\end{equation}
If the risk function of an estimator is smaller than that of another estimator, we say that the former dominates the latter.
Wald's minimum requirement for an estimator is that the estimator is \emph{admissible}, i.e., that it is not dominated. 
Maybe surprisingly, the MLE for regression with the squared loss $L_2$ is \emph{not} admissible! 
This is an extension by \cite{TBC} of an important theorem known as the James-Stein theorem \citep{TBC}.
We leave the original James-Stein theorem as \exo{ex:james-stein}. 
We shall see an even simpler estimator that dominates the MLE in Section~\ref{s:ridge_regression}.

So what should we do, according to Wald, if the MLE is no option? 
It would be natural to find an estimator with as small as possible a risk function.
Unfortunately, the risk function being a function, many pairs of estimators are incomparable. 
Wald might recommend to sum up a risk function by a single number, and look, for instance, for an estimator minimizing the worst-case risk, a so-called \emph{minimax} estimator
\begin{equation}
    \widehat\theta_{\text{minimax}} \in \arg\inf_{\widehat{\theta}} \sup_\theta R(\theta,\widehat{\theta}).
    \label{e:minimax_estimator_for_regression}
\end{equation}
Another solution to sum up the risk function is to integrate it against a measure for which the risk is integrable. 
This leads to an estimator that we call the \emph{Bayes} estimator, anticipating over Chapter~\ref{ch:expected_utility},
\begin{equation}
    \tBayes \in \arg\min_{\widehat{\theta}} \mathbb{E}_\theta R(\theta,\widehat{\theta}) = \mathbb{E}_\theta \mathbb{E}_{\by\vert X,\theta} R(\theta,\widehat{\theta}).
    \label{e:bayes_estimator_for_regression}
\end{equation}
We now have a joint distribution over $\theta,\by$. 
As long as the loss $L(\theta,\widehat{\theta}(\by))$ is integratble w.r.t. this joint, the towering property of the expectation yields
\begin{align}
    \mathbb{E}_\theta \mathbb{E}_{\by\vert X,\theta} L(\theta,\widehat{\theta}) 
    &= \mathbb{E}_\by \mathbb{E}_{\theta\vert X,\by} L(\theta,\widehat{\theta}).
\label{e:reversed_expectation}
\end{align}
Since $\widehat{\theta}$ is only a function $\by$, minimizing \eqref{e:reversed_expectation} boils down to setting 
$$
\tBayes = \tBayes(\by) = \arg\inf_{\widehat{\theta}} \mathbb{E}_{\theta\vert X,\by} L(\theta,\widehat{\theta}).
$$
For the squared loss $L=L_2$, the Bayes estimator is thus the mean of the \emph{posterior} distribution, i.e. 
$
\mathbb{E}_{\theta\vert X,\by} \theta. 
$
For the one-loss $L=L_1$, the Bayes estimator is a generalized median of the same posterior distribution.
In particular, note that the Bayes estimator depends on the loss function, and is not necessarily the posterior expectation.

\section{Savage does linear regression}
\label{ch:bayesian_regression}
% introduce subjective Bayes

\section{Hoerl and Kennard do linear regression}
\label{ch:penalized_regression}
% introduce empirical Bayes
