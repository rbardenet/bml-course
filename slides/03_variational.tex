% TO DO: add citations in thesis work, better research plan

\documentclass[10pt]{beamer}
\usetheme[height=0mm]{Rochester}
\usecolortheme{}
\usefonttheme[onlylarge]{structurebold}
\setbeamerfont*{frametitle}{size=\normalsize,series=\bfseries}
\setbeamertemplate{navigation symbols}{}
\setbeamercovered{dynamic}
\setbeamertemplate{itemize item}[triangle]
\setbeamertemplate{itemize subitem}[triangle]

%  use Darmstadt if commenting the next line
\usepackage[footheight=1em]{beamerthemeboxes}
\usepackage[natbib=true,backend=biber,citestyle=authoryear]{biblatex}
\bibliography{../bib/stats,../bib/learning}

\usepackage{amsmath,amssymb,amsthm,bbm}             % AMS Math
\usepackage[utf8]{inputenc}
\usepackage[english]{babel}
\usepackage{xcolor}
\usepackage{graphicx}
\usepackage{subfigure}
\usepackage{hyperref}
\usepackage{mathtools}
\usepackage{mathrsfs}
% % notation
\DeclareMathOperator*{\argmax}{arg\,max}
\DeclareMathOperator*{\argmin}{arg\,min}
\newcommand{\cA}{\mathcal{A}}
\newcommand{\cS}{\mathcal{S}}
\newcommand{\cF}{\mathcal{F}}
\newcommand{\cX}{\mathcal{X}}
\newcommand{\cY}{\mathcal{Y}}
\newcommand{\cZ}{\mathcal{Z}}
\newcommand{\cN}{\mathcal{N}}
\newcommand{\cO}{\mathcal{O}}
\renewcommand{\leq}{\leqslant}
\renewcommand{\geq}{\geqslant}
\renewcommand{\phi}{\varphi}
\renewcommand{\epsilon}{\varepsilon}
\renewcommand{\d}{ {\rm d}}
\renewcommand{\emptyset}{\varnothing}

\usepackage{tikz}
\usetikzlibrary{bayesnet}
\usetikzlibrary{arrows}
\usepackage{url}

\def\figdir{Figures}

\setbeamertemplate{footline}[frame number]
\newcommand\un[1]{\textcolor{magenta}{#1}}
\newcommand\unn[1]{\textcolor{blue}{#1}}
\newcommand\unnn[1]{\textcolor{red}{#1}}
\newcommand\unnnn[1]{\textcolor{orange}{#1}}
\renewcommand\d{\mathrm{d}}
\newcommand\argmin{\arg\min}
\def\cO{\mathcal{O}}
\def\blank{\vspace{.5\textheight}}
\definecolor{vert}{rgb}{0.47, 0.67, 0.19}

\AtBeginSection[] % Do nothing for \section*
{
\begin{frame}
\frametitle{Outline}
\tableofcontents[currentsection]
\end{frame}
}

\newcommand{\backupbegin}{
   \newcounter{framenumberappendix}
   \setcounter{framenumberappendix}{\value{framenumber}}
}
\newcommand{\backupend}{
   \addtocounter{framenumberappendix}{-\value{framenumber}}
   \addtocounter{framenumber}{\value{framenumberappendix}}
}
\makeatletter

\title[Bayesian ML: Bayesics]{BML lecture \#3: variational Bayes}
\subtitle{\url{http://github.com/rbardenet/bml-course}}
\author[Rémi Bardenet (CNRS \& Univ. Lille)] % (optional, for multiple authors)
{Rémi Bardenet}
\institute[] % (optional)
{
  CNRS \& CRIStAL, Univ. Lille, France\\
\vspace{1cm}
\includegraphics[width=0.15\textwidth]{/Users/rbardenet/Work/Tex/PosterImages/logoCNRS.pdf}
\qquad \includegraphics[width=0.4\textwidth]{/Users/rbardenet/Work/Tex/PosterImages/cristalLogo.pdf}
}
\date{}

\begin{document}
\begin{frame}
\maketitle
\end{frame}

\begin{frame}
\frametitle{Outline}
\tableofcontents
\end{frame}

%%%%%%%%%%%%%%%%%%%%%%
\section{Introduction}
%%%%%%%%%%%%%%%%%%%%%%

\begin{frame}{What comes to $your$ mind when you hear "variational inference"?}
\end{frame}

\section{Variational inference}
\begin{frame}{When MCMC is intractable, variational inference comes to help}
\begin{block}{Turning integration into optimization over measures}
  Variational Bayesian inference (VB) consists in approximating
  $$ 
    \int f(\theta)\pi(\theta)\d\theta \approx \int f(\theta)q^\star(\theta)\d\theta
  $$
  with \un{$ q^\star \in \argmin_{q\in \mathcal Q} \text{distance}(\pi, q)$}. 
  Often we take
  $$
    \text{distance}(\pi, q) = \text{KL}( q, \pi) := \int q(\theta) \log \frac{q(\theta)}{\pi(\theta)}\d\theta.
  $$
  for computational convenience.
\end{block}
\blank
\end{frame}

\begin{frame}{But remember we can only evaluate $\pi_u = Z\pi$...}
\begin{itemize}
\item Show that $J(q) := \int q(\theta) \log \frac{q(\theta)}{\pi_u(\theta)}\d\theta = \text{KL}(q, \pi) - \log Z.$
\blank
\item In particular, $L(q) = -J(q) \leq \log Z$. For
$$
\pi_u(\theta) = p(\text{data}\vert\theta) p(\theta),
$$
$L(q)$ is thus \un{a lower bound for the evidence} $p(\text{data})$ (ELBO).
\end{itemize}
\end{frame}

\begin{frame}{Choosing the approximating family $\mathcal Q$}
\begin{itemize}
  \item The most common approach is the mean-field approximation
  $$ \mathcal Q = \{\theta\mapsto \prod_{d=1}^D q_d(\theta_d)\}.$$
  \item Include all variables over which you integrate, e.g.
  \vfill
  \vfill
  $$
  q(\theta, z_{1:n}) = \prod_{d=1}^D q_d(\theta_d) \prod_{i=1}^N q_i(z_i).
  $$
\end{itemize}
\begin{itemize}
  \item Try to keep some dependence if it is key in your application.
  \item If your original model has NEF conditionals, \un{coordinate-wise maximization of $q\mapsto L(q)$ is easy}.
\end{itemize}
\end{frame}

\begin{frame}{Mean-field yields closed-form updates}
\end{frame}
\section{Back to LDA}
\begin{frame}{Back to LDA 1/2}
  \footnotesize
\begin{align*}
  \log &p(y, z, \pi, B)\\
  &=\sum_{i=1}^N \left[\log p(\pi_i\vert \alpha) + \sum_{\ell=1}^{L_i} \Big(\un{\log p(z_{i\ell}\vert\pi_i)} + \unn{\log p(y_{i\ell}\vert z_{i\ell}, B)\Big)}\right] + \unnnn{p(B\vert \gamma)}\\
  &\propto \sum_{i=1}^N \left[\sum_{k=1}^K \alpha_k \log \pi_{ik} + \sum_{\ell=1}^{L_i} \left(\un{\sum_{k=1}^K 1_{z_{i\ell}=k}\log\pi_{_{ik}}} + \unn{\sum_{v=1}^V \sum_{k=1}^K 1_{y_{i\ell}=v}1_{z_{i\ell}=k} \log b_{kv}}\right)\right] \\
  &~~~~~+ \unnnn{\sum_{k=1}^K \sum_{v=1}^V \gamma_k\log b_{kv}}.
\end{align*}
\vfill
\vspace{1cm}
\begin{block}{Lemma (exercise)}
Let $\Psi(\cdot) := \Gamma'(\cdot)/\Gamma(\cdot)$ be the digamma function. Then
$$
\mathbb{E}_{\text{Dir}(\theta\vert\eta)} \log \theta_i = \Psi(\eta_i) - \Psi(\Vert \eta\Vert_1) =: \Psi_i(\eta).
$$
\end{block}
\vfill
\end{frame}

\begin{frame}{VB for LDA: singling out $\pi_i$}
  \fontsize{6pt}{7pt}\selectfont
\begin{align*}
\log &p(y, z, \pi, B)\\
     & \propto \sum_{i=1}^N \left[\sum_{k=1}^K \alpha_k \log \pi_{ik} + \sum_{\ell=1}^{L_i} \left(\un{\sum_{k=1}^K 1_{z_{i\ell}=k}\log\pi_{_{ik}}} + \unn{\sum_{v=1}^V \sum_{k=1}^K 1_{y_{i\ell}=v}1_{z_{i\ell}=k} \log b_{kv}}\right)\right] 
    + \unnnn{\sum_{k=1}^K \sum_{v=1}^V \gamma_k\log b_{kv}}.
\end{align*}
\vfill
\blank
\end{frame}

\begin{frame}{VB for LDA: singling out $z_{i\ell}$}
  \fontsize{6pt}{7pt}\selectfont
\begin{align*}
\log &p(y, z, \pi, B)\\
     & \propto \sum_{i=1}^N \left[\sum_{k=1}^K \alpha_k \log \pi_{ik} + \sum_{\ell=1}^{L_i} \left(\un{\sum_{k=1}^K 1_{z_{i\ell}=k}\log\pi_{_{ik}}} + \unn{\sum_{v=1}^V \sum_{k=1}^K 1_{y_{i\ell}=v}1_{z_{i\ell}=k} \log b_{kv}}\right)\right] 
    + \unnnn{\sum_{k=1}^K \sum_{v=1}^V \gamma_k\log b_{kv}}.
\end{align*}
  \vfill
  \blank
\end{frame}

\begin{frame}{VB for LDA: singling out $B_{k:}$}
  \fontsize{6pt}{7pt}\selectfont
\begin{align*}
\log &p(y, z, \pi, B)\\
     & \propto \sum_{i=1}^N \left[\sum_{k=1}^K \alpha_k \log \pi_{ik} + \sum_{\ell=1}^{L_i} \left(\un{\sum_{k=1}^K 1_{z_{i\ell}=k}\log\pi_{_{ik}}} + \unn{\sum_{v=1}^V \sum_{k=1}^K 1_{y_{i\ell}=v}1_{z_{i\ell}=k} \log b_{kv}}\right)\right] 
    + \unnnn{\sum_{k=1}^K \sum_{v=1}^V \gamma_v\log b_{kv}}.
  \end{align*}
  \vfill
  \blank
\end{frame}


\begin{frame}{Using counts keeps space and time complexity low}
\begin{itemize}
  \item Storing $\tilde z_{i\ell k}$ requires $\cO (N K\sum_i L_i )$ space. \un{In practice, one works with (sparse) count data}
  $$
  n_{iv} = \text{number of times word $v$ appears in document $i$},
  $$
  and variables $c_{ivk}$, thus reducing storage costs (and the dimension of the underlying integral!) to $\cO(NVK)$.
\blank
\end{itemize}
\end{frame}

\begin{frame}{Gradient-based VB}
  \vfill
  \blank
\end{frame}

\begin{frame}{VB for deep networks 1/2}
  \vfill
  \blank
\end{frame}

\begin{frame}{VB for deep networks 2/2}
  \vfill
  \blank
\end{frame}

\begin{frame}{Lots of variants of VB exist \citep{Mur12}}
  \begin{itemize}
    \item For hidden variable models, \un{EM} is VB with $$q(z,\theta) = \pi(z\vert \theta)\delta_{\tilde\theta}(\theta).$$
    \item \un{Variational EM} is VB with $$q(z,\theta) = q(z)\delta_{\tilde\theta}(\theta).$$
    \item VB for any PGMs with NEF arrows is \un{variational message passing}.
    \item Rather approximating $$\pi(\theta) \approx \prod_{f=1}^F q_f(\theta)$$ leads to \un{expectation propagation}.
    \item These days, \un{ADVI with stochastic gradients} is the default VI choice in probabilistic programming software like PyMC3, Stan, or PyRo.
  \end{itemize}
\end{frame}

\section{Generalizing VB}
\begin{frame}{``Optimization-centric'' Bayesian inference\footcite{KnJeDa22}}
  Given a loss $\ell$, a divergence $D$, and a set of distributions $\mathcal{Q}$, consider\begin{equation}
  \label{e:GVI}
  \tag{$P(\ell, D, \mathcal{Q})$}
  q^\star \in \arg\min_{q\in \mathcal{Q}} \mathbb{E}_q \sum_{i=1}^N \ell(\theta, x_i) + D(q(\theta)\d\theta, p(\theta)d\theta).
\end{equation}
\blank
\end{frame}

\begin{frame}{Conclusion on VB}
  \begin{itemize}
    \item Computationally attractive alternative to MCMC.
    \item Lots of open questions on connecting VB and to the original SEU problem: is VB justified in itself or is it simply a computationally convenient backup option?
    \item Many partial answers, e.g. try to use VB in importance sampling, optimization-centric viewpoint.
    \item More meaningful alternatives to the KL: maximum mean discrepancy, etc.
  \end{itemize}
\end{frame}



\section*{References}
\setbeamertemplate{bibliography item}[text]%,
\begin{frame}[allowframebreaks]
\frametitle{References}
\small
\printbibliography
\normalsize
\end{frame}

\end{document}
