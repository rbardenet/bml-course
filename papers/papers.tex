\documentclass[12pt]{article}%
\usepackage{amssymb}
\usepackage{amsfonts}
\usepackage{amsmath}
\usepackage{amsthm}
\usepackage{fancyhdr}
\usepackage{parskip}

\usepackage{url}
\usepackage[nohead]{geometry}
\geometry{left=1in,right=1in,top=1.00in,bottom=1.0in}
\date{}

\usepackage{graphicx}%
\usepackage[colorlinks=True,urlcolor=blue]{hyperref}

%\usepackage{natbib} % natbib is essentially incompatible with multibib
%\setcitestyle{numbers}
%\renewcommand{\cite}{\citet}
\usepackage[resetlabels,labeled]{multibib}
% biblio commands
\newcites{A}{\large Lecture on Bayesics}
\newcites{B}{\large Lecture on MCMC}
\newcites{C}{\large Lecture on variational inference}
\newcites{D}{\large Lecture on foundations}
\newcites{E}{\large Lecture on Bayesian nonparametrics}
\newcites{F}{\large Lecture on Bayesian deep learning}




\begin{document}

\title{Bayesian ML 2021-22: project topics}
\author{Julyan Arbel, R\'emi Bardenet}
\maketitle

\section{Nature of the project}
Students should form groups of two, each group undertaking one project. We suggest in Section \ref{s:papers} a few scientific papers that can each lead to a project, but you can choose another paper, subject to our approval.

For the paper your group will have chosen, you should: (1) explain the theoretical, computational and/or empirical methods, (2) emphasize the main points of the paper, and (3) apply it to real data of your choice when applicable. Bonus points will be considered if you are creative and add something insightful that is not in the original paper: this can be a theoretical point, an illustrative experiment, etc. The whole point is to read the paper with a critical mind.

\section{Assignment of papers}
As a first step, we ask each group to fill the spreadsheet at
\begin{center}
   \href{https://lite.framacalc.org/cdlr2k2rpq-9sey}{https://lite.framacalc.org/cdlr2k2rpq-9sey}
 \end{center}
with the title of the paper, a link to it (if available), and the composition of the group.
We ask that you fill in the form {\bf before February 17 23:59}. 
%Note that we are aware that some papers are longer or harder than others, and we will take this into account.
%By that time, you will have had an outline of the last two courses, so that you can make your choices with enough information. Then, we will solve the \href{https://en.wikipedia.org/wiki/Assignment_problem}{assignment problem} to find an optimal matching of papers and groups, and will make the result known to all as soon as possible on the \href{Github}{https://github.io/bml-course} page of the course. We will allow up to two groups per paper, but in that case we expect of course the deliverables to be significantly different.

\section{Format of the deliverable}
You can use either Python or R for the programming part. Please have each group send
\begin{itemize}
\item one report as a pdf ($\leq 5$ pages) in the \href{https://www.overleaf.com/latex/templates/neurips-2020/mnshsmqkjsqz}{NeurIPS template},
\item the link to a \href{https://github.com/}{GitHub} or  \href{https://about.gitlab.com/}{GitLab} repository containing your code, a Jupyter notebook to demo the code, and a readme file with instructions to (compile/install and) run the code.
\end{itemize} to \href{mailto:julyan.arbel@inria.fr,remi.bardenet@gmail.com}{both teachers}\footnote{if the above link is broken, this means: \href{mailto:julyan.arbel@inria.fr}{julyan.arbel@inria.fr}, and \href{mailto:remi.bardenet@gmail.com}{remi.bardenet@gmail.com}} {\bf no later than March 21 23:59}. There will be no deadline extension.

The last step of your project will be a presentation in front of the class on March 24.

\section{Proposed papers}
\label{s:papers}

\bibliographystyleA{plain}
\bibliographystyleB{plain}
\bibliographystyleC{plain}
\bibliographystyleD{plain}
\bibliographystyleE{plain}
\bibliographystyleF{plain}

\nociteA{*}
\nociteB{*}
\nociteC{*}
\nociteD{*}
\nociteE{*}
\nociteF{*}
\bibliographyA{part1.bib}
\bibliographyB{part2.bib}
\bibliographyC{part3.bib}
\bibliographyD{part4.bib}
\bibliographyE{part5.bib}
\bibliographyF{part6.bib}
\end{document}
